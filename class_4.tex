
% Start with Kleene
% copied from previous class
\section{Lecture 4}
\subsection{Proofs by Induction}

\subsubsection{Inductively Defined Operations on Strings}

\begin{definition}[Concatenation]
  \label{cl4:def:concat}
For all $y \in \Sigma^*$, we have that $\epsilon;y = y$.
For all $a \in \Sigma$, and $x;y \in \Sigma^*$, we have that $(a\cdot x); y = a \cdot (x;y)$.
\end{definition}

\begin{remark} Binary operation ``$\cdot$'' is taken from the inductive definition of $\Sigma^*$ whereas ``$;$'' is newly defined.
\end{remark}

\begin{theorem}[Associativity]
\label{cl4:thrm:assoc}
$\forall x,y,z \in \Sigma^*$ we have that $(x;y);z = x;(y;z)$.
    
\end{theorem}
\begin{proof} Consider arbitrary $\hat{y}, \hat{z} \in \Sigma^*$. The goal is to show that for all $x \in \Sigma^*$, it holds that $(x; \hat{y});\hat{z} = x ; (\hat{y}; \hat{z})$. We use induction on $x$ and we have the following cases:
    \begin{itemize}
        \item \textbf{$x = \epsilon$}. We get that $(\epsilon; \hat{y}); \hat{z} = \epsilon; (\hat{y}; \hat{z})$. Using Definition $\ref{cl4:def:concat}$ we get $\hat{y};\hat{z} = \hat{y};\hat{z}$.
        \item \textbf{$x = \hat{a}\cdot \hat{u}$} for some $\hat a \in \Sigma$ and $\hat{u} \in \Sigma^*$, we have that the induction hypothesis is $(\hat{u}; \hat{y}); \hat{z} = \hat{u};(\hat{y}; \hat{z})$ and the goal is to show that 
        \begin{equation}
            ((\hat{a}\cdot \hat{u}); \hat{y}); \hat{z} = (\hat{a} \cdot \hat{u}); (\hat{y}; \hat{z}).
        \end{equation}  
        If we apply Definition \ref{cl4:def:concat} we get
        \begin{equation}
            (\hat{a}\cdot (\hat{u}; \hat{y})); z = \hat{a}\cdot(\hat{u}; (\hat{y}; \hat{z})).
            \label{eq:1}
        \end{equation}
        Applying Definition \ref{cl4:def:concat} to the left side and the induction hypothesis to the right side of equation \ref{eq:1} we get
        \begin{equation}
            \hat{a} \cdot ((\hat{u}; \hat{y}); \hat{z}) = \hat{a} \cdot ((\hat{u}; \hat{y}); \hat{z}). 
        \end{equation}
    \end{itemize}
\end{proof}

\begin{definition}[$\reverseA$]
Let $\reverseA(\epsilon) = \epsilon$.
For all $a \in \Sigma$ and $x \in \Sigma^*$, we have that $\reverseA(a\cdot x) = \reverseA(x); a$.
    \label{cl4:def:rev1}
\end{definition}


\begin{definition}[$\reverseB$]
\label{cl4:def:rev2}
For all $y\in \Sigma^*$, we have that $\reverseB(\epsilon, y) = y$.
For all $a\in \Sigma$ and $x, y\in \Sigma^*$, we have that $\reverseB(a \cdot x, y) = \reverseB(x, a\cdot y)$.
\end{definition}



\begin{theorem}
For all $x \in \Sigma^*$, $\reverseA(x) = \reverseB(x, \epsilon)$.
\end{theorem}

\begin{proof} Proof by induction on $x$ would stuck, so we aim for a stronger statement.

\begin{lemma}
For all $x, y \in \Sigma^*$, $\reverseA(x) ; y = \reverseB(x, y)$.
\end{lemma}
\begin{proof} We prove it by induction on $x$.
    \begin{itemize}
        \item For the base case we have to prove that for $x=\epsilon$; $\reverseA(\epsilon);y =\reverseB(\epsilon, y)$. Using Definitions $\ref{cl4:def:rev1}$ and $\ref{cl4:def:rev2}$ we get that $y=y$.
        \item As the induction hypothesis we assume that for all $y \in \Sigma^*$ and $x = \hat{a}\cdot \hat{u}$, where $\hat{a} \in \Sigma$ and $\hat{u}\in\Sigma^*$, it holds that $\reverseA(\hat{u}); y = \reverseB(\hat{u}, y)$. Then we have to show that 
        \begin{equation}
            \reverseA(\hat{a}\cdot \hat{u}); y = \reverseB(\hat{a}\cdot\hat{u}, y).
            \label{eq:rev_ind}
        \end{equation}
        \begin{itemize}
            \item Consider an arbitrary $\hat{y}$. If we apply Definitions $\ref{cl4:def:rev1}$ and $\ref{cl4:def:rev2}$ to the left and the right side respectively we get
            \begin{equation}
                (\reverseA(\hat{u}); \hat{a});\hat{y} = \reverseB(\hat{u}, \hat{a}\cdot \hat{y}).
            \end{equation}
            \item Then we apply Theorem \ref{cl4:thrm:assoc} to the left side and get
            \begin{equation}
                \reverseA(\hat{u}); (\hat{a}; \hat{y})= \reverseB(\hat{u}, \hat{a}\cdot \hat{y}).
            \end{equation}
            Finally, we apply the induction hypothesis to the left side of our previous equation and get
            \begin{equation}
                \reverseB(\hat{u}, \hat{a}\cdot \hat{y}) = \reverseB(\hat{u}, \hat{a}\cdot \hat{y}).
            \end{equation}
        \end{itemize}
    \end{itemize}
\end{proof}
To finish the proof of the theorem, we substitute $y := \epsilon$.
\end{proof}

\subsubsection{Well-founded Induction}

\begin{definition}[Well-founded] A binary relation $\prec$ on a set $A$ is  \underline{well-founded} if there is no infinite descending sequence $x_0 \succ x_1 \succ x_2 \succ \ldots$ on $A$.
\end{definition}

\begin{example} Two examples are $<$ over the set of natural numbers and "shorter than" on $\Sigma^*$.
\end{example}


\begin{remark}[Well-founded induction]
 In order to show $\forall x\in A$, $G(x)$ for well-founded $\prec$. Consider an arbitrary $\hat{x} \in A$.
Assume $\forall y \prec \hat{x}$, $G(y)$.
Show $G(\hat{x})$.
\end{remark}

\begin{definition}[$\odd$ and $\even$] For all $x \in \Sigma^*$ and $a \in \Sigma$, we have that $\odd(\epsilon) = \epsilon$, $\odd(a\cdot x) = a\cdot \even(x)$, $\even(\epsilon) = \epsilon$ and $\even(a\cdot x) = \odd(x)$.
\end{definition}


\begin{definition}[$\merges$] Let $\Sigma^*$ be an alphabet with linear operator $\leq$. Then for all $x,y \in \Sigma^*$ and $a,b \in \Sigma$ we have that: (i) $\merges(\epsilon, x) = x$, (ii) $\merges(y, \epsilon) = y$, and (iii) $\merges(a\cdot x, b\cdot y) = a\cdot \merges(x, b\cdot y)$ if $a \leq b$, otherwise $b\cdot \merges(a\cdot x, y)$.
\end{definition}

\begin{definition}[$\mergesort$] For all $x \in \Sigma^*$, we have that $\mergesort = \merges(\mergesort(\odd(x))$ and  $\mergesort(\even(x)))$
\end{definition}

\begin{definition}[$\sorted$] We define a predicate $\sorted$ such that for all $a \in \Sigma$, we have that $\sorted(\epsilon)$, $\sorted(a\cdot\epsilon)$ and for all $x\in \Sigma^*$, we have that $\sorted(a\cdot b\cdot x)$ iff $a\leq b$ and $\sorted(b\cdot x)$.
\end{definition}

\begin{lemma}
\label{cl4:lemma:merge-sort}
For all $x,y \in \Sigma^*$, if $\sorted(x)$ and $\sorted(y)$, then $\sorted(\merges(x,y))$.
\end{lemma}


\begin{theorem}[Merge-Sort]
\label{cl4:thrm:merge-sort}
For all $x\in \Sigma^*$, we have that $\sorted(\mergesort(x))$.
\end{theorem}


\begin{theorem}[Permutation]
\label{cl4:thrm:permutation}
For all $x \in \Sigma^*$, $\mergesort(x)$ is a permutation of $x$.
\end{theorem}


\begin{exercise}
 (i) Proof Lemma~\ref{cl4:lemma:merge-sort} and Theorem~\ref{cl4:thrm:merge-sort}. (ii) Write a definition of permutation. 
 (iii) Proof Theorem~\ref{cl4:thrm:permutation}
\end{exercise}


\begin{definition}
\label{cl4:def:hm1}
For all $x$ and $y$, we have $x > y$ iff $\parent(x,y)$ or there exists z such that $\parent(x,z)$ and $z > y$.
\end{definition}
\begin{definition} 
\label{cl4:def:hm2}
For all x and y, we have $x > y$ iff $\parent(x,y)$ or there exists z such that $x > z$ and $\parent(z,y)$.
\end{definition}


\begin{axiom}For all $x$, we have that $\human(x)$ implies $\neg \monkey(x)$ and $\monkey(x)$ implies $\neg \human(x)$.
\end{axiom}

\begin{axiom} 
$<$ is well-founded.
\end{axiom}



\begin{theorem}
\label{cl4:thrm:hm}
Provide a proof of the following using Def.~\ref{cl4:def:hm1} and then using Def.~\ref{cl4:def:hm2}. If there exist $x$ and $y$ such that $x > y$ and $\monkey(x)$ and $\human(y)$, then there exist $x$ and $y$ such that $\parent(x,y)$ and $\monkey(x)$ and $\human(y)$.
\end{theorem}

\begin{exercise}
Proof Theorem~\ref{cl4:thrm:hm}.
\end{exercise}
